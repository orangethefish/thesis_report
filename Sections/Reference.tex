\newpage
\addcontentsline{toc}{section}{Tài liệu tham khảo}
\renewcommand\refname{Tài liệu tham khảo}
\begin{thebibliography}{3}
    \item Deng J., Zhang S., and Ma J. (2023). Self-Attention-Based Deep Convolution LSTM Framework for Sensor-Based Badminton Activity Recognition. Sensors, 23(20), 8373. \href{https://doi.org/10.3390/s23208373}{Link tham khảo}
    \item Ordóñez F. and Roggen D. (2016). Deep convolutional and LSTM recurrent neural networks for multimodal wearable activity recognition. Sensors, 16(1), 115. \href{https://doi.org/10.3390/s16010115}{Link tham khảo}
    \item Vaswani A., Shazeer N., Parmar N., et al. (2017). Attention is all you need. arXiv.org, <https://arxiv.org/abs/1706.03762>. \href{https://arxiv.org/abs/1706.03762}{Link tham khảo}
    \item Deb S.D., Jha R.K., Kumar R., et al. (2023). CoVSeverity-Net: an efficient deep learning model for COVID-19 severity estimation from Chest X-Ray images. Research on Biomedical Engineering, 39(1). \href{https://doi.org/10.1007/s42662-022-00377-1}{Link tham khảo}
    \item Fu F., Chen J., Zhang J. và cộng sự. (2024). Are Synthetic Time-series Data Really not as Good as Real Data?. arXiv.org, <https://arxiv.org/abs/2402.00607>. \href{https://arxiv.org/abs/2402.00607}{Link tham khảo}
    \item Ioffe S. và Szegedy C. (2015). Batch normalization: Accelerating deep network training by reducing internal covariate shift. arXiv.org, <https://arxiv.org/abs/1502.03167>.
    \item LouisFoucard/MC_DCNN: Multi channel deep convolutional neural network for time series classification. GitHub, <https://github.com/LouisFoucard/MC_DCNN?tab=readme-ov-file>.
    \item 1.Faisal M.A.A. (2022). ASL-Sensor-Dataglove-Dataset.zip. , accessed: 07/12/2024.
    \item Yoon J., Jarrett D., và Van Der Schaar M. (2019). Time-series generative adversarial networks. Neural Information Processing Systems, 32.
    \item Bishop C. M. (1995). Neural Networks for Pattern Recognition, Nhà in Đại học Oxford. ISBN 0-19-853864-2
    \item Microchip Technology. (truy cập lần cuối tháng 12/2023). "ATmega328P Microcontroller." Truy cập từ: \href{https://www.microchip.com/en-us/product/ATmega328P}{Link tham khảo} 
    \item Arduino Kit Vietnam. (Ngày đăng 12/06/2023). "Hướng dẫn sử dụng cảm biến uốn cong (Flex Sensor) với Arduino." Truy cập từ: \href{https://arduinokit.vn/huong-dan-su-dung-cam-bien-uon-cong-flex-sensor-voi-arduino/}{Link tham khảo} 
    \item Pham, D. K. (Ngày đăng 22/04/2019). "Lý thuyết về mạng LSTM." Truy cập từ: \href{https://phamdinhkhanh.github.io/2019/04/22/Ly_thuyet_ve_mang_LSTM.html?fbclid=IwAR0rBn2SL1vPNYmLgixLra-BoMMb6f98mhPrMxv_BhX13_9sK3_4da1vN6c}{Link tham khảo}
    \item Brownlee, J. (Ngày đăng 28/08/2020). "How to Develop RNN Models for Human Activity Recognition (Time Series Classification)." Truy cập từ: \href{https://machinelearningmastery.com/how-to-develop-rnn-models-for-human-activity-recognition-time-series-classification/}{Link tham khảo}  
    
\end{thebibliography}