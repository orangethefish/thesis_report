\section{Kết luận}
\subsection{Kết quả đạt được}

\indent Nhóm đã xây dựng được một hệ thống có thể chuyển đổi cử chỉ của một người khuyết tật thành giọng nói thông qua việc thu thập dữ liệu từ 5 cảm biến flex sensor và cảm biến gia tốc góc. Găng tay được thiết kế để thu thập dữ liệu từ các cử chỉ của người dùng và gửi dữ liệu đến máy tính thông qua giao tiếp Bluetooth. Dữ liệu sau đó được xử lý và đưa vào mô hình học để dự đoán cử chỉ và chuyển đổi thành giọng nói. Găng tay có vai trò quan trọng trong việc thu thập dữ liệu và dự đoán cử chỉ của người dùng.


\indent Trong giai đoạn 2 này với các cải thiện mà nhóm đã thực hiện, nhìn chung, việc khả năng dự đoán của thiết bị có sự cải thiện rõ rệt nhất sau khi thay đổi mô hình để huấn luyện dữ liệu với khác biệt thực tế có thể lên đến 20\%. Các kỹ thuật khác cũng giúp mô hình cải thiện độ chính xác nhưng không nhiều như việc thay đổi mô hình. Việc kết hợp mô hình mới với dữ liệu đã làm giàu cũng giúp mô hình cải thiện độ chính xác. Tuy nhiên, có một số kỹ thuật không mang lại hiệu quả như mong đợi.

\subsection{Hạn chế của hệ thống}
\indent Một trong những hạn chế lớn nhất của hệ thống là hiện tại việc chuyển đổi từ cử chỉ thành giọng nói chỉ đang được ghi nhận bằng một tay phải. Thực tế, việc giao tiếp bằng ngôn ngữ ký hiệu không chỉ dừng lại ở hai tay mà còn bao gồm cả cơ thể. 

\indent Hơn nữa, việc sử dụng máy tính làm trung gian cho việc nhận và dự đoán dữ liệu cử chỉ cũng là một hạn chế lớn. Điều này làm việc áp dụng hệ thống vào thực tế trở nên không khả thi.

\subsection{Hướng cải tiến và phát triển}
\indent Với những hạn chế và thách thức mà nhóm vừa nêu, có một số hướng phát triển mà nhóm có thể hướng đến:
\begin{itemize}
    \item Xây dựng thêm 1 găng tay cho tay trái để thu thập dữ liệu từ cả hai tay và có thể kết hợp sử dụng thêm cảm biến gia tốc cho găng tay. Việc các cử chỉ tăng lên thì cần có thêm dữ liệu từ một tay mới hay một cảm biến khác để đảm bảo mô hình vẫn có thể dự đoán chính xác.
    \item Xây dựng một hệ thống nhúng với pin nhỏ gọn để có thể gắn trên người và có thể xử lý dữ liệu cử chỉ. Điều này giúp loại bỏ việc sử dụng máy tính. Mọi dữ liệu thu thập từ găng tay sẽ được xử lý và chuyển đổi cử chỉ đó thành âm thanh tương ứng. Việc xây dựng hệ thống nhúng với pin như vậy cũng đề ra thách thức về việc tiết kiệm năng lượng và xử lý dữ liệu một cách hiệu quả. Việc đề ra một mô hình mới tuy tiêu hao nhiều điện năng nhưng vẫn có thể đảm bảo chất lượng đàu ra hay việc chọn loại pin nào thiết kê hệ thống như thế nào để vẫn có thể đảm bảo tính nhỏ gọn và nhẹ nhàng cũng là một thách thức lớn 
\end{itemize}