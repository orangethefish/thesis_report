\section{Kết luận}

\subsection{Kết quả đạt được}

\indent Trong giai đoạn 1 này, nhóm đã thành công trong việc triển khai một sản phẩm tích hợp flex sensor, kết hợp với Arduino Nano, để thu thập dữ liệu từ chuyển động của các ngón tay. Sản phẩm này đóng vai trò quan trọng trong việc cung cấp dữ liệu đầu vào cho quá trình huấn luyện và dự đoán của mô hình LSTM.

\indent Bên cạnh đó, nhóm đã hiện thực một mô hình LSTM có khả năng dự đoán các hành động được định nghĩa trước, thông qua quá trình huấn luyện trên tập dữ liệu chứa các mẫu dữ liệu liên quan đến các hành động đó. Điều này đặt nền tảng cho tính linh hoạt và độ chính xác của hệ thống, giúp nó có khả năng hiểu và phản ứng đáng tin cậy đối với các cử chỉ của người sử dụng.

\subsection{Giới hạn của hệ thống}

\indent Mặc dù hệ thống đã đạt được sự thành công cơ bản, tuy nhiên, chúng tôi không thể phủ nhận sự tồn tại của một số giới hạn quan trọng. Một trong những thách thức lớn nhất là khả năng mô hình LSTM trực tiếp dự đoán từ dữ liệu thu thập được. Điều này xuất phát từ sự phức tạp và đa dạng của dữ liệu, đòi hỏi quá trình xử lý để tạo ra đầu ra dự đoán chính xác và hiệu quả.

\indent Quá trình xử lý dữ liệu kéo dài thời gian giữa việc thu thập dữ liệu và quá trình dự đoán. Điều này có thể tạo ra một khoảng thời gian trễ không mong muốn, ảnh hưởng đến tính linh hoạt và phản ứng nhanh chóng được mong đợi từ hệ thống.

\subsection{Định hướng cho giai đoạn tiếp theo}

\indent Đối mặt với những thách thức và giới hạn đã được đề cập trước đó, mục tiêu quan trọng trong giai đoạn tiếp theo của chúng tôi là nâng cao mô hình LSTM để có khả năng dự đoán trực tiếp từ dữ liệu thu thập được. Điều này đòi hỏi sự tập trung vào việc tối ưu hóa và cải thiện quá trình xử lý dữ liệu, nhằm giảm thiểu độ trễ và tăng cường tính linh hoạt của hệ thống.

\indent Ngoài ra, chúng tôi đặt ra mục tiêu thiết kế một mô hình nhà thông minh toàn diện, trong đó các thiết bị có thể được điều khiển bằng cử chỉ của người dùng thông qua flex sensor và mô hình LSTM. Điều này mở ra cơ hội để tối ưu hóa không chỉ việc dự đoán mà còn quá trình tương tác và điều khiển, tạo ra một trải nghiệm người dùng linh hoạt và hiệu quả.