\section{Giới thiệu đề tài}
\subsection{Giới thiệu}
% \indent Giao tiếp là một phần không thể thiếu trong cuộc sống của con người. Giao tiếp giúp con người có thể dẽ dàng trao đổi thông tin, bày tỏ cảm xúc, từ đó gây dựng nên kết nối xã hội giữa người với người. Tuy nhiên, đối với những người có khiếm khuyêt như câm hoặc điếc thì việc giao tiếp là một trở ngại cực kì lớn đối với họ. Theo một thóng kê của Tổ chức Y tế thế giới(WHO) vào năm 2023, trên thế giới hiện nay có khoảng 70 triệu người khiếm thính, phần lớn trong số đố cũng ghi nhận tình trạng mất khả năng nói và khoảng hơn 430 triệu người có vấn đề về thính giác.\\
% \indent Phương tiện giao tiếp với những người có khiếm khuyết về thính giác cũng như giọng nói được sử dụng chủ yếu hiện nay là ngôn ngữ ký hiệu hay còn gọi là thủ ngữ. Dù vậy, tỉ lệ người bình thường có thể sử dụng thành thạo hiện nay là rất nhỏ. Điều đó đặt ra rất nhiều thách thức trong việc giao tiếp giữa người bình thường và người gặp khiếm khuyết. Chính vì thế, nhóm chọn đề tài "Thiết bị hỗ trợ chuyển đổi ngôn ngữ ký hiệu thành giọng nói sử dụng flex sensors và học máy" nhằm mục đích thiết kế và hiện thực một hệ thống có thể chuyển ngôn ngữ ký hiệu thành giọng nói để hỗ trợ trong việc giao tiếp của người gặp khiếm khuyết.
\indent Ngôn ngữ ký hiệu là công cụ giao tiếp quan trọng của cộng đồng người khiếm thính và câm, giúp họ truyền tải thông tin và ý tưởng trong cuộc sống hàng ngày. Tuy nhiên, một rào cản lớn là phần lớn những người không quen thuộc với ngôn ngữ ký hiệu gặp khó khăn trong việc hiểu và tương tác với nhóm đối tượng này. Điều này gây ra sự bất tiện và làm giảm khả năng hòa nhập xã hội của người khuyết tật.

\indent Đề tài "Thiết bị chuyển đổi ngôn ngữ ký hiệu sang giọng nói sử dụng Flex Sensors kết hợp với học máy" hướng đến việc phát triển một giải pháp công nghệ nhằm thu hẹp khoảng cách này. Thiết bị được thiết kế với các flex sensors để ghi nhận chuyển động của bàn tay khi thực hiện ký hiệu. Thông qua việc kết hợp với các thuật toán học máy, hệ thống có khả năng nhận diện ngôn ngữ ký hiệu và chuyển đổi chúng thành giọng nói một cách chính xác và tự nhiên.

\indent Ứng dụng này không chỉ giúp người khiếm thính giao tiếp dễ dàng hơn mà còn mở ra nhiều cơ hội để họ hòa nhập vào các hoạt động xã hội và nghề nghiệp. Với tiềm năng ứng dụng rộng rãi, thiết bị này hứa hẹn sẽ mang lại lợi ích thiết thực không chỉ trong lĩnh vực giao tiếp mà còn trong giáo dục, y tế và các ngành dịch vụ.
\subsection{Mục tiêu}
\indent Mục tiêu của đề tài là xây dựng một hệ thống thông minh có khả năng nhận diện chính xác cử chỉ tay thông qua dữ liệu thu thập từ flex sensors và cảm biến con quay hồi chuyển, sau đó chuyển đổi các cử chỉ này thành giọng nói một cách tự nhiên và mượt mà. Hệ thống không chỉ tập trung vào tính chính xác và hiệu quả trong nhận diện mà còn hướng đến việc nâng cao khả năng ứng dụng thực tiễn, hỗ trợ giao tiếp cho cộng đồng người khiếm thính và câm, góp phần cải thiện chất lượng cuộc sống và tăng cường sự hòa nhập xã hội.
\subsection{Phạm vi của đề tài}
\indent Ngôn ngữ ký hiệu bao gồm nhiều cử chỉ đa dạng và phức tạp, đòi hỏi lượng dữ liệu lớn để phân tích và xử lý một cách hiệu quả. Do đó, phạm vi nghiên cứu của đề tài sẽ được giới hạn thành hai phần chính nhằm đảm bảo tính khả thi và tiến độ thực hiện.
\begin{itemize}
    \item \textbf{Thu thập dữ liệu cử chỉ:} Tập trung vào thiết kế và hiện thực phần cứng, sử dụng các flex sensors và cảm biến con quay gia tốc để thu thập dữ liệu liên quan đến các cử chỉ tay. Phần này sẽ đảm bảo thu thập dữ liệu chính xác và đầy đủ để phục vụ quá trình xử lý.
    \item \textbf{Phân tích, xử lý dữ liệu:} Triển khai và tối ưu hóa mô hình học máy để phân tích dữ liệu thu thập được, từ đó nhận diện và dự đoán chính xác các cử chỉ tay. Phần này là nền tảng để hệ thống chuyển đổi cử chỉ thành giọng nói.
\end{itemize}


\subsection{Ý nghĩa thực tiễn}
\indent Đề tài mang lại ý nghĩa thực tiễn to lớn trong việc cải thiện khả năng giao tiếp giữa người khiếm thính hoặc câm và những người không biết ngôn ngữ ký hiệu, góp phần nâng cao chất lượng cuộc sống và khả năng hòa nhập xã hội của họ. Sản phẩm của đề tài sở hữu nhiều ưu điểm nổi bật khi áp dụng vào thực tế, cụ thể như sau:
\begin{itemize}
    \item \textbf{Giao tiếp dễ dàng hơn:} Thiết bị cho phép người khiếm thính thực hiện các ký hiệu tay, sau đó chuyển đổi chúng thành giọng nói một cách tự động, giúp họ giao tiếp thuận tiện với mọi người mà không cần người phiên dịch.
    \item \textbf{Tính tiện lợi:} Nhờ thiết kế nhỏ gọn và khả năng sử dụng linh hoạt, thiết bị có thể được mang theo và sử dụng mọi lúc, mọi nơi, giúp việc giao tiếp trở nên tự nhiên hơn trong các tình huống hàng ngày.
    \item \textbf{Khả năng kết nối:} Thiết bị có thể tích hợp với các nền tảng công nghệ hiện đại như điện thoại di động hoặc máy tính, không chỉ hỗ trợ phát giọng nói mà còn có khả năng chuyển đổi ký hiệu thành văn bản, phục vụ cho các hình thức giao tiếp như tin nhắn hay email.
    \item \textbf{Tính đa dụng:} Hệ thống học máy cho phép thiết bị nhận diện và học hỏi nhiều cử chỉ từ đơn giản đến phức tạp, đáp ứng nhu cầu sử dụng của đa dạng người dùng và phù hợp với nhiều ngữ cảnh khác nhau.
\end{itemize}

\indent Với những tính năng này, thiết bị không chỉ mở ra cơ hội giao tiếp mới cho người khiếm thính và câm, mà còn thúc đẩy sự ứng dụng công nghệ trong cuộc sống. Đề tài góp phần xây dựng một xã hội công bằng, nơi mọi người đều có cơ hội giao tiếp, phát triển và hòa nhập, đồng thời khẳng định vai trò quan trọng của công nghệ trong việc cải thiện đời sống con người.