\section{Giới thiệu đề tài}
\subsection{Giới thiệu}
\indent Giao tiếp là một phần không thể thiếu trong cuộc sống của con người. Giao tiếp giúp con người có thể dẽ dàng trao đổi thông tin, bày tỏ cảm xúc, từ đó gây dựng nên kết nối xã hội giữa người với người. Tuy nhiên, đối với những người có khiếm khuyêt như câm hoặc điếc thì việc giao tiếp là một trở ngại cực kì lớn đối với họ. Theo một thóng kê của Tổ chức Y tế thế giới(WHO) vào năm 2023, trên thế giới hiện nay có khoảng 70 triệu người khiếm thính, phần lớn trong số đố cũng ghi nhận tình trạng mất khả năng nói và khoảng hơn 430 triệu người có vấn đề về thính giác.\\
\indent Phương tiện giao tiếp với những người có khiếm khuyết về thính giác cũng như giọng nói được sử dụng chủ yếu hiện nay là ngôn ngữ ký hiệu hay còn gọi là thủ ngữ. Dù vậy, tỉ lệ người bình thường có thể sử dụng thành thạo hiện nay là rất nhỏ. Điều đó đặt ra rất nhiều thách thức trong việc giao tiếp giữa người bình thường và người gặp khiếm khuyết. Chính vì thế, nhóm chọn đề tài "Thiết bị hỗ trợ chuyển đổi ngôn ngữ ký hiệu thành giọng nói sử dụng flex sensors và học máy" nhằm mục đích thiết kế và hiện thực một hệ thống có thể chuyển ngôn ngữ ký hiệu thành giọng nói để hỗ trợ trong việc giao tiếp của người gặp khiếm khuyết.
\subsection{Mục tiêu}
\indent Mục tiêu của đề tài là hiện thực một hệ thống có thể nhận diện được cử chỉ từ dữ liệu thu thập bằng cảm biến uốn cong và cảm biến con quay hồi chuyển và chuyển đổi cử chỉ đó thành giọng nói.
\subsection{Phạm vi của đề tài}
\indent Ngôn ngữ ký hiệu là một ngôn ngữ với nhiều cử chỉ đa dạng và phức tạp, yêu cầu một lượng dữ liệu rất lớn để có thể phân tích và xử lý. Cho nên nhóm sẽ giới hạn phạm vi của đề tài thành hai phần chính để đảm bảo tiến độ cũng như tính khả thi của đề tài. Hai phần trọng tâm của đề tài bao gồm:
\begin{itemize}
    \item \textbf{Thu thập dữ liệu cử chỉ:} thiết  kế, hiện thực phần cứng để thu thâp dữ liệu cử chỉ bằng  các cảm biến uốn cong và cảm biến con quay gia tốc.
    \item \textbf{Phân tích, xử lý dữ liệu:} xây dưng mô hình học máy để đưa ra kết quả dữ đoán cử chỉ từ những dữ liệu đã thu thập được.
\end{itemize}


\subsection{Ý nghĩa thực tiễn}
\indent Với đề tài này, nhóm hy vọng có thể  có ý nghĩa thực tiễn rất lớn trong việc cải thiện khả năng giao tiếp giữa người khiếm thính hoặc câm và người không biết ngôn ngữ ký hiệu, từ đó giúp cải thiện đời sống của họ. 
\indent Sản phẩm của đề tài có một số ưu điểm nổi bật khi áp dụng vào thực tế như sau
\begin{itemize}
    \item \textbf{Giao tiếp dễ dàng hơn:} Thiết bị này giúp người khiếm thính có thể giao tiếp với những người không biết ngôn ngữ ký hiệu một cách dễ dàng hơn. Họ chỉ cần thực hiện các ký hiệu, và thiết bị sẽ chuyển đổi chúng thành giọng nói.
    \item \textbf{Tính tiện lợi:} Thiết bị có thể được sử dụng mọi lúc mọi nơi, giúp người khiếm thính có thể giao tiếp một cách tự nhiên và thoải mái hơn trong cuộc sống hàng ngày
    \item \textbf{Khả năng kết nối:} Thiết bị có thể kết nối với các thiết bị khác như điện thoại di động hoặc máy tính, giúp việc chuyển đổi và phát giọng nói trở nên dễ dàng và thuận tiện hơn. Không những thế còn có thể chuyển đổi thành văn bản để sử dụng trong đối thoại bằng tin nhắn hoặc email.
    \item \textbf{Tính đa dụng:} Với việc sử dụng học máy, thiết bị có thể học và nhận biết một loạt các ký hiệu từ đơn giản đến phức tạp, giúp nó có thể phục vụ một lượng lớn người dùng và áp dụng cho nhiều ngữ cảnh khác nhau.
\end{itemize}

\indent Như vậy, đề tài này không chỉ mở ra cơ hội mới cho người khiếm thính hoặc câm trong việc giao tiếp mà còn đẩy mạnh tiến trình ứng dụng công nghệ vào cuộc sống, nhằm tạo ra một xã hội công bằng và tiến bộ hơn.